\documentclass[12pt, a4paper]{article}

% Including essential packages for formatting and structure
\usepackage[utf8]{inputenc}
\usepackage[T1]{fontenc}
\usepackage{geometry}
\usepackage{graphicx}
\usepackage{booktabs}
\usepackage{amsmath}
\usepackage{hyperref}
\usepackage{url}
\usepackage{fancyhdr}
\usepackage{titling}
\usepackage[vietnamese]{babel}

% Configure URL line breaks
\Urlmuskip=0mu plus 1mu

% Configuring page geometry for proper margins
\geometry{margin=1in}

% Setting up headers and footers
\pagestyle{fancy}
\fancyhf{}

\title{Classification of Audio Embeddings}
\author{Turkey Detection Team}
\date{May 25, 2025}

\fancyhead[L]{\leftmark}
\fancyhead[R]{\thepage}
\fancyfoot[C]{\thetitle}

% Configuring title and author for the cover page
\pretitle{\begin{center}\LARGE\bfseries}
\posttitle{\par\end{center}\vskip 0.5em}
\preauthor{\begin{center}\large}
\postauthor{\end{center}}

% Setting the font to Latin Modern for consistency
\usepackage{lmodern}

\begin{document}

% Creating the cover page
\begin{titlepage}
    \centering
    \vspace*{2cm}

    \vspace{1.5cm}

    \vspace{1.5cm}
    
    % Report title
    \Huge \textbf{Classification of Audio Embeddings} \\
    
    \vspace{1cm}
    
    % Subject
    \Large \textbf{Subject: Introduction to Machine Learning} \\
        
    \vspace{0.5cm}

    \includegraphics[scale=.40]{img/hcmus-logo.png}\\[1cm]
    
    \vspace{0.5cm}
    
    % Supervised teachers
    \large \textbf{Supervised by: Bùi Duy Đăng, Nguyễn Thanh Tình} \\

    \vspace{0.5cm}

    % Location
    \large \textbf{Location: Thành phố Hồ Chí Minh, Việt Nam} \\

    \vspace{0.5cm}
    
    % School name
    \large \textbf{Trường Đại học Khoa Học Tự Nhiên, tp Hồ Chí Minh} \\
    
    \vspace{1cm}
    
    % Group information
    \large \textbf{Prepared by: Group 30} \\

    % Date
    \small \text{Date: May 25, 2025} \\
    
    \vfill
    
\end{titlepage}

\newpage

% Group Information Section
\section{Group Information}
This project was completed by the group \textbf{Group 30}. The team members are listed below:

\begin{table}[h]
    \centering
    \begin{tabular}{ll}
        \toprule
        \textbf{Name} & \textbf{Student ID} \\
        \midrule
        Bùi Kim Phúc & 21120112 \\
        Lê Hoàng Sơn & 21120127 \\
        \bottomrule
    \end{tabular}
    \caption{Group Members}
    \label{tab:group_members}
\end{table}

% About the Project Section
\section{About the Project}
The objective of this project is to develop a machine learning model to classify audio clips based on the presence of turkey sounds. The input data consists of audio embeddings, each represented as a matrix of shape $[10, 128]$, where 10 is the number of frames and 128 is the dimensionality of each frame’s feature vector. These embeddings are extracted from audio clips and provided in a JSON file (\texttt{train.json}). The task is a binary classification problem, where the model predicts whether an audio clip contains turkey sounds (\texttt{is\_turkey = 1}) or not (\texttt{is\_turkey = 0}). The classification is performed using a logistic regression model implemented with Scikit-learn, leveraging the flattened embeddings (size 1280) as input features.
Finally, the model's performance is evaluated on a test set, and the results are submitted in a CSV file (\texttt{submission.csv}) for Kaggle competition.
% Data Preprocessing Section
\section{Data Preprocessing}
The dataset was sourced from a JSON file (\texttt{train.json}) containing audio embeddings and binary labels. The preprocessing steps included:

\begin{itemize}
    \item Loading the JSON data into a pandas DataFrame.
    \item Extracting \texttt{audio\_embedding} (shape [10, 128]) and \texttt{is\_turkey} (binary labels: 0 or 1).
    \item Flattening each audio embedding to a 1D array of size 1280 (10 $\times$ 128).
    \item Padding or truncating embeddings to ensure a consistent size.
    \item The data contains a total of 1195 samples and is splitted into 836 training samples (70\%), 179 validation samples (15\%), and 180 test samples (15\%) with shuffling to ensure randomness.
\end{itemize}

% Model Description Section
\section{Model Description}
The models used for this project are a logistic regression classifier and a random forest classifier, both implemented using Scikit-learn. Key details for each model are as follows:

\begin{itemize}
    \item \textbf{Logistic Regression}:
        \begin{itemize}
            \item \textbf{Algorithm}: Logistic regression with the \texttt{liblinear} solver, suitable for small datasets.
            \item \textbf{Input Features}: Flattened audio embeddings of size 1280 (10 frames $\times$ 128 dimensions).
            \item \textbf{Output}: Binary classification (0 or 1) for the \texttt{is\_turkey} label.
            \item \textbf{Hyperparameters}: C=1.0, max\_iter=1000, random\_state=42.
        \end{itemize}
    \item \textbf{Random Forest}:
        \begin{itemize}
            \item \textbf{Algorithm}: Random forest classifier, an ensemble method using multiple decision trees.
            \item \textbf{Input Features}: Flattened audio embeddings of size 1280 (10 frames $\times$ 128 dimensions).
            \item \textbf{Output}: Binary classification (0 or 1) for the \texttt{is\_turkey} label.
            \item \textbf{Hyperparameters}: n\_estimators=100, max\_depth=None, random\_state=42.
        \end{itemize}
\end{itemize}

% Platform Section
\section{Platform}
The project was developed and executed on the following platform:

\begin{itemize}
    \item \textbf{Google Colab}: The primary development environment, providing a cloud-based Jupyter notebook interface with GPU support for efficient model training and evaluation.
    \item \textbf{Google Drive}: Used for storing and accessing the dataset (\texttt{train.json}) and saving the submission file (\texttt{submission.csv}).
    \item \textbf{Libraries}: Scikit-learn for model implementation, pandas for data processing, NumPy for numerical operations, and other standard Python libraries.
    \item \textbf{Latex}: The report is formatted using LaTeX for professional presentation.
\end{itemize}

% Training Section
\section{Training}
The training process involved the following steps for both the logistic regression and random forest models:

\begin{itemize}
    \item \textbf{Data Loading}: The JSON dataset was loaded using pandas and processed to extract features and labels.
    \item \textbf{Data Splitting}: The dataset was split into:
        \begin{itemize}
            \item Training set: 70\% of the data, including 836 samples.
            \item Validation set: 15\% of the data, including 179 samples.
            \item Test set: 15\% of the data, including 180 samples.
        \end{itemize}
        Splitting was performed using Scikit-learn's \texttt{train\_test\_split} with a random state of 42 for reproducibility.
    \item \textbf{Model Training}:
        \begin{itemize}
            \item \textbf{Logistic Regression}: The model was trained on the training set using the \texttt{fit} method with the \texttt{liblinear} solver.
            \item \textbf{Random Forest}: The model was trained on the training set using the \texttt{fit} method with an ensemble of decision trees.
        \end{itemize}
    \item \textbf{Validation}: Both models were evaluated on the validation set to tune hyperparameters and assess performance.
\end{itemize}

% Evaluation Metrics Section
\section{Evaluation Metrics}
The model’s performance was evaluated using the following metrics:

\begin{itemize}
    \item \textbf{Accuracy}: The proportion of correct predictions on the validation and test sets.
\end{itemize}

% Accuracy Report Section
\section{Accuracy Report}
The performance of both models is reported below:

\begin{itemize}
    \item \textbf{Logistic Regression}:
        \begin{itemize}
            \item \textbf{Validation Accuracy}: 0.9497.
            \item \textbf{Test Accuracy}: 0.9167.
        \end{itemize}
    \item \textbf{Random Forest}:
        \begin{itemize}
            \item \textbf{Validation Accuracy}: 0.8939.
            \item \textbf{Test Accuracy}: 0.9000.
        \end{itemize}
\end{itemize}

% Submission File Section
\section{Submission}
Submission files include:
\begin{itemize}
    \item \texttt{submission.csv}: Our best result achieves a test accuracy of 93.347\% using the logistic regression model on Kaggle.
    \item \texttt{Turkey30\_Scikit\_learn.ipynb}: The Jupyter notebook containing the complete code for data preprocessing, model training, evaluation, and submission generation.
    \item \texttt{This report}: The LaTeX report detailing the project, methodology, and results.
\end{itemize}

% Conclusion Section
\section{Conclusion}

\indent In this project, we successfully implemented a machine learning model to classify audio embeddings for the presence of turkey sounds. The logistic regression model achieved a validation accuracy of 94.97\% and a test accuracy of 91.67\%, while the random forest model achieved a validation accuracy of 89.39\% and a test accuracy of 90.00\%.

\indent Although these models are quite simple, they demonstrate the feasibility of using audio embeddings for classification tasks.

\indent The logistic regression model outperformed the random forest model in terms of accuracy, indicating that a linear approach was effective for this dataset. Future work could explore more complex models like deep learning Neural Networks for potentially better performance.

% References Section
\section{References}
\begin{itemize}
    \item Scikit-learn Documentation: \url{https://scikit-learn.org/stable/}
    \item Pandas Documentation: \url{https://pandas.pydata.org/}
    \item Google Colab: \url{https://colab.research.google.com/}
    \item Kaggle competition page: \\
        \url{https://www.kaggle.com/competitions/introduction-to-machine-learning-project-cq-24-2}
\end{itemize}

\section{AI Usage}
The project utilized AI tools including GPT-4, Grok, and Claude Sonnet 3.7 for various tasks such as:

\begin{itemize}
    \item Code syntax support and debugging.
    \item Report helping and formatting in LaTeX assistance.
\end{itemize}

We acknowledge the use of AI tools in enhancing our productivity and code quality, while ensuring that the core logic and implementation were developed by the team.\\
\indent Chat conversations with AI can be found here:
\begin{itemize}
    \item \url{https://grok.com/share/bGVnYWN5_161d9f2e-a240-48ec-a2ac-be01d0881371}
\end{itemize}

\end{document}