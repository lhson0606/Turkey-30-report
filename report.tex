% Setting up the document class with standard formatting
\documentclass[12pt, a4paper]{article}

% Including essential packages for formatting and structure
\usepackage[utf8]{inputenc}
\usepackage[T1]{fontenc}
\usepackage{geometry}
\usepackage{graphicx}
\usepackage{booktabs}
\usepackage{amsmath}
\usepackage{hyperref}
\usepackage{fancyhdr}
\usepackage{titling}

% Configuring page geometry for proper margins
\geometry{margin=1in}

% Setting up headers and footers
\pagestyle{fancy}
\fancyhf{}

\title{Classification of Audio Embeddings Using Logistic Regression}
\author{Turkey Detection Team}
\date{May 25, 2025}

\fancyhead[L]{\leftmark}
\fancyhead[R]{\thepage}
\fancyfoot[C]{\thetitle}

% Configuring title and author for the cover page
\pretitle{\begin{center}\LARGE\bfseries}
\posttitle{\par\end{center}\vskip 0.5em}
\preauthor{\begin{center}\large}
\postauthor{\end{center}}

% Setting the font to Latin Modern for consistency
\usepackage{lmodern}

\begin{document}

% Creating the cover page
\begin{titlepage}
    \centering
    \vspace*{2cm}
    
    % Placeholder for school logo
    % Replace with \includegraphics[width=0.5\textwidth]{logo.png} when adding the logo
    \fbox{\parbox{0.5\textwidth}{\centering \textbf{[School Logo Placeholder]}}}

    \vspace{1.5cm}
    
    % Report title
    \Huge \textbf{Classification of Audio Embeddings Using Logistic Regression} \\
    
    \vspace{1cm}
    
    % Subject
    \Large \textbf{Subject: Machine Learning Project} \\
    
    \vspace{0.5cm}
    
    % Date
    \large \textbf{Date: May 25, 2025} \\
    
    \vspace{0.5cm}
    
    % Supervised teachers
    \large \textbf{Supervised by: [Teacher Name(s)]} \\
    
    \vspace{0.5cm}
    
    % Location
    \large \textbf{Location: [City, Country]} \\
    
    \vspace{0.5cm}
    
    % School name
    \large \textbf{[School Name]} \\
    
    \vspace{2cm}
    
    % Group information
    \large \textbf{Prepared by: [Group Name]} \\
    
    \vfill
    
\end{titlepage}

\newpage

% Group Information Section
\section{Group Information}
This project was completed by the group \textbf{[Group Name]}. The team members are listed below:

\begin{table}[h]
    \centering
    \begin{tabular}{ll}
        \toprule
        \textbf{Name} & \textbf{Student ID} \\
        \midrule
        Bùi Kim Phúc & 21120112 \\
        Mary Johnson & 20210456 \\
        Alex Williams & 20210789 \\
        % Add more rows as needed
        \bottomrule
    \end{tabular}
    \caption{Group Members}
    \label{tab:group_members}
\end{table}

% About the Project Section
\section{About the Project}
The objective of this project is to develop a machine learning model to classify audio clips based on the presence of turkey sounds. The input data consists of audio embeddings, each represented as a matrix of shape $[10, 128]$, where 10 is the number of frames and 128 is the dimensionality of each frame’s feature vector. These embeddings are extracted from audio clips and provided in a JSON file (\texttt{train.json}). The task is a binary classification problem, where the model predicts whether an audio clip contains turkey sounds (\texttt{is\_turkey = 1}) or not (\texttt{is\_turkey = 0}). The classification is performed using a logistic regression model implemented with Scikit-learn, leveraging the flattened embeddings (size 1280) as input features.

% Data Preprocessing Section
\section{Data Preprocessing}
The dataset was sourced from a JSON file (\texttt{train.json}) containing audio embeddings and binary labels. The preprocessing steps included:

\begin{itemize}
    \item Loading the JSON data into a pandas DataFrame.
    \item Extracting \texttt{audio\_embedding} (shape [10, 128]) and \texttt{is\_turkey} (binary labels: 0 or 1).
    \item Flattening each audio embedding to a 1D array of size 1280 (10 $\times$ 128).
    \item Padding or truncating embeddings to ensure a consistent size.
    \item Splitting the data into training (70\%), validation (15\%), and test (15\%) sets using stratified splitting to maintain class distribution.
\end{itemize}

% Model Description Section
\section{Model Description}
The models used for this project are a logistic regression classifier and a random forest classifier, both implemented using Scikit-learn. Key details for each model are as follows:

\begin{itemize}
    \item \textbf{Logistic Regression}:
        \begin{itemize}
            \item \textbf{Algorithm}: Logistic regression with the \texttt{liblinear} solver, suitable for small datasets.
            \item \textbf{Input Features}: Flattened audio embeddings of size 1280 (10 frames $\times$ 128 dimensions).
            \item \textbf{Output}: Binary classification (0 or 1) for the \texttt{is\_turkey} label.
            \item \textbf{Hyperparameters}: [Insert hyperparameters here, e.g., C=1.0, max\_iter=100].
        \end{itemize}
    \item \textbf{Random Forest}:
        \begin{itemize}
            \item \textbf{Algorithm}: Random forest classifier, an ensemble method using multiple decision trees.
            \item \textbf{Input Features}: Flattened audio embeddings of size 1280 (10 frames $\times$ 128 dimensions).
            \item \textbf{Output}: Binary classification (0 or 1) for the \texttt{is\_turkey} label.
            \item \textbf{Hyperparameters}: [Insert hyperparameters here, e.g., n\_estimators=100, max\_depth=None, random\_state=42].
        \end{itemize}
\end{itemize}

% Platform Section
\section{Platform}
The project was developed and executed on the following platform:

\begin{itemize}
    \item \textbf{Google Colab}: insert here
    \item \textbf{Google Drive}: Used for storing and accessing the dataset (\texttt{train.json}) and saving the submission file (\texttt{submission.csv}).
    \item \textbf{Libraries}: Scikit-learn for model implementation, pandas for data processing, NumPy for numerical operations, and other standard Python libraries.
    \item \textbf{Latex}: The report is formatted using LaTeX for professional presentation.
\end{itemize}

% Training Section
\section{Training}
The training process involved the following steps for both the logistic regression and random forest models:

\begin{itemize}
    \item \textbf{Data Loading}: The JSON dataset was loaded using pandas and processed to extract features and labels.
    \item \textbf{Data Splitting}: The dataset was split into:
        \begin{itemize}
            \item Training set: 70\% of the data.
            \item Validation set: 15\% of the data.
            \item Test set: 15\% of the data.
        \end{itemize}
        Splitting was performed using Scikit-learn's \texttt{train\_test\_split} with a random state of 42 for reproducibility.
    \item \textbf{Model Training}:
        \begin{itemize}
            \item \textbf{Logistic Regression}: The model was trained on the training set using the \texttt{fit} method with the \texttt{liblinear} solver.
            \item \textbf{Random Forest}: The model was trained on the training set using the \texttt{fit} method with an ensemble of decision trees.
        \end{itemize}
    \item \textbf{Validation}: Both models were evaluated on the validation set to tune hyperparameters and assess performance.
\end{itemize}

% Evaluation Metrics Section
\section{Evaluation Metrics}
The model’s performance was evaluated using the following metrics:

\begin{itemize}
    \item \textbf{Accuracy}: The proportion of correct predictions on the validation and test sets.
\end{itemize}

% Accuracy Report Section
\section{Accuracy Report}
The performance of both models is reported below:

\begin{itemize}
    \item \textbf{Logistic Regression}:
        \begin{itemize}
            \item \textbf{Validation Accuracy}: [Insert validation accuracy here, e.g., 0.XXXX].
            \item \textbf{Test Accuracy}: [Insert test accuracy here, e.g., 0.XXXX].
        \end{itemize}
    \item \textbf{Random Forest}:
        \begin{itemize}
            \item \textbf{Validation Accuracy}: [0.8939].
            \item \textbf{Test Accuracy}: [0.9000].
        \end{itemize}
\end{itemize}

% Submission File Section
\section{Submission File}
(\texttt{submission.csv})

% Conclusion Section
\section{Conclusion}
[Insert your conclusion here, summarizing the project outcomes, model performance, challenges faced, and potential improvements.]

% References Section
\section{References}
\begin{itemize}
    \item Scikit-learn Documentation: \url{https://scikit-learn.org/stable/}
    \item Pandas Documentation: \url{https://pandas.pydata.org/}
    \item Google Colab: \url{https://colab.research.google.com/}
\end{itemize}

\end{document}